\documentclass[12pt]{amsart}

\usepackage{a4wide}

\usepackage{fancyhdr}
\pagestyle{fancy}
\lhead{}
\chead{}
\rhead{}
\lfoot{}
\cfoot{\today}
\rfoot{\thepage}

\usepackage{algorithm2e}
\usepackage{graphicx}
\usepackage[parfill]{parskip}
\usepackage{appendix}
\usepackage{amsmath}
\usepackage{multicol}

\renewcommand{\labelitemi}{-}
\renewcommand{\labelitemii}{-}
\renewcommand{\thefootnote}{\roman{footnote}}

\setcounter{secnumdepth}{10}
\setcounter{tocdepth}{3}

\title{Trader Game}
\author{Ethel Bardsley, Joseph Slade and Thomas Wood}

\begin{document}

\maketitle

\section{Introduction}
  % Brief rundown of the spec
The purpose of this exercise was to gain experience with web programming, 
with running server and client side testing,  and with use of a database, 
all for the purpose of making a multi user program. Specifically,  
this program was a space trading game.  Whilst we stuck with the essence
of this program,  we made a few changes in areas in which we felt we could
improve the core game.
\section{Concept}
  % Brief explanation of the game
  The primary idea of the game was to be a trading game,  as in the original 
spec. We felt that this was a good model for a game,  but we decided to make 
some key changes.  
  \begin{itemize}
    \item Crossplatform
      A crossplatform game has several advantages. It can be played by players
      on any machine, increasing potential audience by a reasonable factor. It
      also allows for testing and development to be done on any platform, which
      certainly makes development easier.  

    \item Multiple maps with a means of teleporting between them
      Multiple maps with single points of access allow interesting modifications
      to core gameplay to be made in cetain zones.  Because seperate maps form
      distinct environments,  they can justifiably (in an in-game sense) have
      different enemies,  different items in their shops,  and even allow certain
      things that could not be implemented on a universal scale,  such as free
      for all PvP (see next section), or co-operative PvE (see next section).
      Although we did not have this implemented at the time this report was
      written, we would like to have a working prototype of it for the presentation.

    \item PvE and PvP\footnote{Player versus Environment, Player versus Player} interaction
      The main selling point of all multiplayer games is the fact that the player can
      interact with others.  Player versus Player (PvP) interaction in any game generally 
      consists of either combat or trade.  We chose to implement both,  and players can
      attack each other as they wish.  There is currently nothing to gain
      from combat,  except for preventing other players from logging their character
      in once dead which has its appeal to certain players.  Players cannot trade
      directly, but the game economy is a dynamic one,  and one player purchasing all
      of a resource will cause a shortage of that resource for a period of time.

    \item A dynamic economy where the value of an item is related to its rarity
      To provide some concept of a goal,  and as an alternative to destroying
      other people's characters,  players can buy and sell resources.  The system 
      is not zero sum\footnote{There is not a fixed amount of gold in the economy}, and
      so players have the opportunity to make money. To prevent this aspect of the game
      being about more than making round trips between two points where there 
      is a fixed discrepancy between item values,  item stocks and values fluctuate over time,
      modified by various factors,  such as the map the shop is on,  the amount of stock
      held at a time,  and any other reasonable thing we can think of.


  \end{itemize}

\section{Program Design}
  % How this work
  \subsection{Renderer}
    \subsubsection{Choice of Technology}
      \begin{flushleft}
        There are a variety of technologies for browser-based rendering. The
        most obvious is Adobe Flash, as well Microsoft's Silverlight. However,
        while these are powerful tools with a wide selection of libraries and
        well-trodden paths for development, neither are universally cross
        platform, and Silverlight is not supported by the version of Firefox
        on the test machines.

        For handling the rendering with just HTML and Javascript, one method is
        to render by manipulating the DOM\footnote{Document Object Model}, and
        there are a few libraries for writing renderers using this method.
        However, DOM manipulation can suffer performance issues, particularly
        where there are many elements being drawn.

        Canvas is a relatively new technology, introduced with HTML5. It allows
        for procedurally drawing directly to an image with Javascript, and is
        supported by Gecko, WebKit (including mobile versions) and, from version
        9, Internet Explorer. There aren't many libraries available for it, and
        those that are aren't mature or well documented. However, canvas itself
        has sufficient framework for 2D sprite drawing, and I took this as a
        chance to learn through my own experience how a 2D renderer can be made.
      \end{flushleft}

    \subsubsection{Implementation}
      \begin{flushleft}
        The render initializes by setting up the canvas element and render
        context, loading map data from the server, using that to generate a
        background image and place any scenery sprites (eg, shops). It then
        centers the view, adds an input handler, before starting the render
        loop.

        The field in the background is procedurally generated, being a
        different image each time the page is loaded. Pixels are a random
        color, weighted toward being a light green. Originally, I tried to have
        each pixel in the map generated uniquely, but this was very CPU
        intensive, causing the page to hang, or even crash. To remedy this,
        instead it generates a set of temporary smaller canvas tiles and uses
        those to build the larger background, which is considerably more
        performant.

        Sprite animation currently supports 2 actions, \verb/stand/ and
        \verb/walk/, although is easily extensible to take more as needed. Each
        action has a number of sprites associated with it, which are simply
        stepped through, 1 per redraw, moving the sprite across the screen if
        necessary. Because of it's simplicity, however, attacking motions must
        happen independently in their own render queue to prevent the main
        actor sprite from disappearing.
      \end{flushleft}
  \subsection{Network}
    \begin{flushleft}
      The frontend makes extensive use of the XMLHttpRequest interface provided 
      by the browser. This interface permits additional HTTP requests to be made 
      to the server by the client-side JavaScript to request that an action be 
      recorded in the database, or to receive more data.

      The client to server data encoding is the same as a standard HTML form 
      would be transmitted as, namely the 
      \verb|application/x-www-form-urlencoded| MIME-type. This is reasonably 
      simple to construct with JavaScript, with each field name and value
      string being URL-escaped then concatenated into a string with \verb|=| 
      separating the fields from the values, and \verb|&| being used to separate 
      each individual field.\footnote{See also: W3C HTML 4.01 Specification Section 
      17.13.4} This format is advantageous for receipt on the server-side as PHP 
      automatically parses requests made in this format and makes them available
      to the program.

      The server to client data is encoded using JSON\footnote{JavaScript Object 
      Notation}. This format is advantageous as a transfer format as it 
      concicely encodes complex data-structures into a reasonably human-readable 
      string. Both PHP and JavaScript provide built-in functions to encode and 
      parse this format, respectively.

      We have also implemented a server-push notifications system using the 
      long-polling of multipart HTTP requests. HTTP is generally limited by the 
      fact that requests can only be made in one direction - a client can only 
      request data from a server, the server cannot arbitrarily open a 
      connection to the client to notify it of an event.
      
      Long-polling is one 
      method to overcome this limitation. Long-polling is when a client makes a 
      request to the server to send it data but if the server has no data to 
      send at the time the request is made, the server holds the connection open 
      until it is ready to send some.
      
      Multipart requests permit multiple distinct responses from the server to 
      the client for an individual request. In standard practice, the content of 
      any previous responses is replaced by the content a new incoming response.

      The combination of long-polling with multipart reponses permits 
      server-pushing to be achieved. As an added precaution against underlying 
      connection timeouts, each long-poll request is terminated after 60 seconds 
      and re-initiated.
    \end{flushleft}

  \subsection{Backend}
    \subsubsection{Choice of Technology}
      \begin{flushleft}
        Our choice of server-side technology was decided by a combination of the
        skills the members of the group already had and for the ease of use with
        the department's Apache server. We chose PHP as the backend programming
        language for several reasons, firstly there was a minimal setup time
        required to get code to work with the provided server, secondly the
        language transparently handles many of the complexities of handling
        requests, thirdly PHP is very similar in syntax to C and in OO-style to
        Java, both languages that the group are familiar with, and finally PHP is
        bundled with a large number of interfaces to external libraries.  In
        particular we use the PDO\footnote{PHP Data Objects} class to interface
        with the PostgresSQL database and the SystemV Semaphore library for
        low-overhead message passing between processes.
      \end{flushleft}

    \subsubsection{Implementation}
      \begin{flushleft}
        Our code structure aims to meet the MVC\footnote{Model View Controller} 
        pattern of object-oriented programming, for this we opted to use one third 
        party PHP library, the ToroPHP Framework\footnote{http://www.toroweb.org/, 
        the code is assumed to be freely licenced as all documentation implies so} 
        is a small (150 line) class to route requests to appropriate handler 
        (controller) classes.

        We implemented a basic object-relational mapping class to make database 
        requests as straightforward as possible. There exists an abstract 
        DatabaseRecord class that handles the each of the standard 
        CRUD\footnote{Create, Read, Update, Delete} actions for all database 
        access. Each individual table has its own class subclassed from 
        DatabaseRecord, set on each of these classes are the fields that are 
        contained and associated functions, such as the login function on the 
        Player class.

        One of the model classes does not use the PostgresSQL database as a 
        backend. For the backend to the long-polling message notifications we
        are using Unix SystemV Shared Memory and Message Queues. We opted to use 
        these rather than the database to minimise the overheads that would 
        result from continually polling the database. Instead, the 
        implementation of reading from the Message Queue cause the process to
        become blocked until an appropriate message is placed in the queue.
      \end{flushleft}

\section{Conclusion}
  % Wrap up, reiterate all the above in short
  \begin{flushleft}
    While we did not acomplish everything we set out to do, we were for the
    most part successful, and with extra time could have accomplished more. 
  \end{flushleft}

\end{document}

